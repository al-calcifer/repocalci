\documentclass[12pt,a4paper]{article}
\usepackage[utf8]{inputenc}
\usepackage[T1]{fontenc}
\usepackage{amsmath}
\newcommand{\angstrom}{\textup{\AA}}
\usepackage{amsfonts}
\usepackage{amssymb}
\usepackage{graphicx}
\author{Antonio Lívio}
\begin{document}
\begin{table}[h]
\centering
\caption{Nanotubos Armchair}
\begin{tabular}{|c|c|c|c|c|c|}
\hline
 
\textit{(n,m)} & $d_{t}^{teo}(\angstrom)$ & $d_{t}^{sim}(\angstrom)$ & $E/N(eV)$ \\ 
\hline 

(10,10) & $13.55$ &  & $-7,3332$ \\ 
\hline
 
(11,11) & $14.91$ &  & $-7,3397$ \\ 
\hline
 
(12,12) & $16.27$ &  & $-7,3448$ \\ 
\hline
 
(13,13) & $17.62$ &  & $-7,3487$ \\ 
\hline
 
(14,14) & $18.98$ &  & $-7,3518$ \\ 
\hline
 
(15,15) & $20.33$ &  & $-7,3543$ \\ 
\hline
 
(16,16) & $21.69$ &  & $-7,3564$ \\ 
\hline
 
(17,17) & $23.05$ &  & $-7,3581$ \\ 
\hline
 
(18,18) & $24.40$ &  & $-7,3595$ \\ 
\hline

(19,19) & $25.76$ &  & $-7,3608$ \\ 
\hline

(20,20) & $27.11$ &  & $-7,3618$ \\ 
\hline

(21,21) & $28.47$ &  & $-7,3627$ \\ 
\hline

 
\end{tabular} 
\end{table}
\emph{*N é o número total de atomos em cada estrutura.}
\end{document}